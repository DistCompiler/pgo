PlusCal has many language constructs that are translated to Go in a non-trivial way. These constructs and their translations are described in this section. The \texttt{pgo/datatypes} Go helper library, which contains some useful data structures and helper methods, is also described.

\subsection{Variable declarations}
In addition to the simple variable declaration \verb|var = <val>|, PlusCal supports the declaration \verb|var \in <set>|. This asserts that the initial value of \texttt{var} is an element of \texttt{<set>}. This is translated into a loop in which each set element is assigned to \texttt{var}. This is the same as the behaviour of TLC in model-checking mode, and is the most likely use case.

\noindent
\begin{minipage}[t]{\textwidth}
\begin{lstlisting}[language=pcal]
variables
	S = {1, 3};
	v \in S;
{
	\* algorithm body...
}
\end{lstlisting}
\captionof{figure}{PlusCal}
\end{minipage}

\begin{minipage}[t]{\textwidth}
\begin{lstlisting}[language=golang]
import "pgo/datatypes"

var S datatypes.Set = datatypes.NewSet(1, 3) // PGo inferred type set[int]
var v int // PGo inferred type int

func main() {
	for v_interface := range S.Iter() {
		v = v_interface.(int)
		// algorithm body...
	}
}
\end{lstlisting}
\captionof{figure}{Compiled Go}
\end{minipage}

(The set methods of the \texttt{pgo/datatypes} package are described in~\ref{sec:sets}.)

If no type annotation was provided for a variable, PGo prints a comment which contains the type that it inferred. If the variable needs thread-safe access, then PGo will also print the lock group that the variable belongs to (a description of locking is in~\ref{sec:labels}).

\subsection{Variable assignment}
PlusCal supports multiple variable assignment statements: the statement \texttt{x := a || y := b} evaluates the right-hand sides first, then assigns the values to the left-hand sides. A common use is swapping the variables \texttt{x} and \texttt{y} with the statement \texttt{x := y || y := x}.

Go has a multiple assignment construct, but a PlusCal assignment might not translate to a Go assignment (for example, assigning a value to a particular map key translates to the \texttt{Put} method). The right-hand sides are first evaluated and assigned to temporary variables, then assigned to their corresponding variables.

\noindent
\begin{minipage}[t]{0.45\textwidth}
\begin{lstlisting}[language=pcal]
x := y || y := x || mapping[i] := a
\end{lstlisting}
\captionof{figure}{PlusCal}
\end{minipage}
\hfill
\begin{minipage}[t]{0.45\textwidth}
\begin{lstlisting}[language=golang]
x_new := y
y_new := x
mapping_new := a
x = x_new
y = y_new
mapping.Put(i, a)
\end{lstlisting}
\captionof{figure}{Compiled Go}
\end{minipage}

\subsection{Macros}
PlusCal macros have the same semantics as C/C++ \texttt{\#define} directives. PGo expands the PlusCal macro wherever it occurs.

\noindent
\begin{minipage}[t]{0.45\textwidth}
\begin{lstlisting}[language=pcal]
variables p = 1, q = 2;
macro add(a, b) {
	a := a + b;
}
{
	add(p, q);
	print p;
}
\end{lstlisting}
\captionof{figure}{PlusCal}
\end{minipage}
\hfill
\begin{minipage}[t]{0.45\textwidth}
\begin{lstlisting}[language=golang]
import "fmt"

var p int = 1
var q int = 2

func main() {
	p = p + q
	fmt.Println("%v", p)
}
\end{lstlisting}
\captionof{figure}{Compiled Go}
\end{minipage}

\subsection{Data types}
PlusCal supports maps, sets, and tuples, which are implemented in the \texttt{pgo/datatypes} library.
\subsubsection{Sets}
\label{sec:sets}
PlusCal sets are implemented in Go by the \texttt{datatypes.Set} type. This container is an ordered, thread-safe set which has methods equivalent to PlusCal set operations.

\noindent
\begin{minipage}[t]{0.45\textwidth}
\begin{lstlisting}[language=pcal]
variables
	A = {1, 2, 3};
	B = 3 .. 6; \* the set {3, 4, 5, 6}
	C = A \union B;
	D = SUBSET C; \* powerset of C
{
	print A = C;
}
\end{lstlisting}
\end{minipage}
\hfill
\begin{minipage}[t]{0.45\textwidth}
\begin{lstlisting}[language=golang]
import (
	"fmt"
	"pgo/datatypes"
)

var A datatypes.Set = datatypes.NewSet(1, 2, 3)
var B datatypes.Set = datatypes.Sequence(3, 6)
var C datatypes.Set = A.Union(B)
var D datatypes.Set = C.PowerSet()

func main() {
	fmt.Printf("%v\n", A.Equal(C))
}
\end{lstlisting}
\end{minipage}

PlusCal also supports the typical mathematical set constructor notations, which are offloaded to library methods:

\noindent
\begin{minipage}[t]{\textwidth}
\begin{lstlisting}[language=pcal]
variables
	S = {1, 5, 6};
	T = {2, 3};
	U = {x \in S : x > 3}; \* equivalent to {5, 6}
	V = {x + y : x \in S, y \in T}; \* equivalent to {3, 4, 7, 8, 9}
\* ...
\end{lstlisting}
\captionof{figure}{PlusCal}
\end{minipage}

\noindent
\begin{minipage}[t]{\textwidth}
\begin{lstlisting}[language=golang]
import "pgo/datatypes"

var S datatypes.Set = datatypes.NewSet(1, 5, 6)
var T datatypes.Set = datatypes.NewSet(2, 3)
var U datatypes.Set = datatypes.SetConstructor(func(x int) bool {
		return x > 3
	}, S)
var V datatypes.Set = datatypes.SetImage(func(x int, y int) int {
		return x + y
	}, S, T)
// ...
\end{lstlisting}
\captionof{figure}{Compiled Go}
\end{minipage}

While not as concise as the PlusCal, the output Go code is still readable.

The Cartesian product operator \verb|\X| in PlusCal is not associative. For example,
\begin{itemize}
\item \texttt{<<1, 2, 3>>} is in \verb|Nat \X Nat \X Nat|,

\item \texttt{<< <<1, 2>>, 3>>} is in \verb|(Nat \X Nat) \X Nat|, and

\item \texttt{<<1, <<2, 3>> >>} is in \verb|Nat \X (Nat \X Nat)|,
\end{itemize}
and none of these are equivalent. In Go, \texttt{S.CartesianProduct(T, U, V, ...)} is equivalent to \verb|S \X T \X U \X ...|. PGo composes these function calls to create the parenthesized expressions above.

\noindent
\begin{minipage}[t]{\textwidth}
\begin{lstlisting}[language=pcal]
variables
	S = {1, 2, 3};
	T = S \X S \X S; \* << 1, 3, 2 >> \in T
	U = S \X (S \X S); \* << 1, << 3, 2 >> >> \in U
\end{lstlisting}
\captionof{figure}{PlusCal}
\end{minipage}

\noindent
\begin{minipage}[t]{\textwidth}
\begin{lstlisting}[language=golang]
import "pgo/datatypes"

var S datatypes.Set = datatypes.NewSet(1, 2, 3)
var T datatypes.Set = S.CartesianProduct(S, S)
var U datatypes.Set = S.CartesianProduct(S.CartesianProduct(S))
\end{lstlisting}
\captionof{figure}{Compiled Go}
\end{minipage}

\subsubsection{Maps}
PlusCal ``functions'' are just maps in Go. \texttt{pgo/datatypes} implements an ordered, thread-safe map in Go, which supports using slices and tuples as keys (unlike the builtin Go map). There are several syntaxes to declare maps, which are handled by library methods similar to the set methods.

A PlusCal function can be indexed by multiple indices. This is syntactic sugar for a map indexed by a tuple whose components are the indices.

\noindent
\begin{minipage}[t]{\textwidth}
\begin{lstlisting}[language=pcal]
variables
	S = {1, 2};
	f = [x \in S, y \in S |-> x + y];
	a = f[2, 2]; \* a = 4
\end{lstlisting}
\captionof{figure}{PlusCal}
\end{minipage}

\noindent
\begin{minipage}[t]{\textwidth}
\begin{lstlisting}[language=golang]
import "pgo/datatypes"

var S datatypes.Set = datatypes.NewSet(1, 2)
var f datatypes.Map = datatypes.MapsTo(func(x int, y int) int {
		return x + y
	}, S, S)
var a int = f.Get(datatypes.NewTuple(2, 2))
\end{lstlisting}
\captionof{figure}{Compiled Go}
\end{minipage}

(Behind the scenes, the \texttt{datatypes.Map} uses tuples as keys.)

\subsubsection{Tuples}
PlusCal tuples are used in several different contexts, so variables involving tuples must be annotated with their types. Tuples can store homogeneous data, in which case they correspond to Go slices. Tuple components may be of different types, which correspond to \texttt{datatypes.Tuple}s. Most often in multiprocess algorithms, tuples are used for communication, in which case \texttt{datatypes.Tuple}s are used. PlusCal tuples are 1-indexed, but Go tuples and slices are 0-indexed, so 1 is subtracted from all indices in Go.

\noindent
\begin{minipage}[t]{\textwidth}
\begin{lstlisting}[language=pcal]
(*	@PGo{ var slice []string }@PGo
	@PGo{ var tup tuple[int,string] }@PGo *)
variables
	slice = << "a", "b", "c" >>;
	tup = << 1, "a" >>;
{
	print slice[2]; \* "b"
}
\end{lstlisting}
\captionof{figure}{PlusCal}
\end{minipage}

\noindent
\begin{minipage}[t]{\textwidth}
\begin{lstlisting}[language=golang]
import (
	"fmt"
	"pgo/datatypes"
)

var slice []string = []string{"a", "b", "c"}
var tup datatypes.Tuple = datatypes.NewTuple(1, "a")

func main() {
	fmt.Printf("%v", slice[2 - 1])
}
\end{lstlisting}
\captionof{figure}{Compiled Go}
\end{minipage}

\subsection{Predicate operations}
PlusCal supports the mathematical quantifiers $\forall$ and $\exists$. PGo compiles these to the library methods \texttt{ForAll} and \texttt{Exists}.

\noindent
\begin{minipage}[t]{\textwidth}
\begin{lstlisting}[language=pcal]
variables
	S = {1, 2, 3};
	T = {4, 5, 6};
	b1 = \E x \in S : x > 2 \* TRUE
	b2 = \A x \in S, y \in T : x + y > 6 \* FALSE
\end{lstlisting}
\captionof{figure}{PlusCal}
\end{minipage}

\noindent
\begin{minipage}[t]{\textwidth}
\begin{lstlisting}[language=golang]
import "pgo/datatypes"

var S datatypes.Set = datatypes.NewSet(1, 2, 3)
var T datatypes.Set = datatypes.NewSet(4, 5, 6)
var b1 bool = datatypes.Exists(func(x int) bool {
		return x > 2
	}, S)
var b2 bool = datatypes.ForAll(func(x int, y int) bool {
		return x + y > 6
	}, S, T)
\end{lstlisting}
\captionof{figure}{Compiled Go}
\end{minipage}

Related is Hilbert's $\varepsilon$ operator (called \texttt{CHOOSE} in PlusCal). In PlusCal, \verb|CHOOSE x \in S : P(x)| is defined to be some arbitrary value in \texttt{S} such that \texttt{P(x)} is true. If there is no such \texttt{x}, the expression is unspecified. Furthermore, \texttt{CHOOSE} is deterministic and always selects the same \texttt{x} given equal sets and equivalent predicates. The Go equivalent function \texttt{datatypes.Choose} selects the least element that satisfies the predicate. This guarantees the determinism that the operator provides, and is the same as TLC's behaviour. If no element satisfies the predicate \texttt{P}, \texttt{datatypes.Choose} panics, similar to TLC returning an error when it encounters unspecified behaviour.

\noindent
\begin{minipage}[t]{\textwidth}
\begin{lstlisting}[language=pcal]
variables
	S = {1, 2, 3};
	a = CHOOSE x \in S : x >= 1
	b = CHOOSE y \in S : y < 0 \* TLC error
\end{lstlisting}
\captionof{figure}{PlusCal}
\end{minipage}

\noindent
\begin{minipage}[t]{\textwidth}
\begin{lstlisting}[language=golang]
import "pgo/datatypes"

var S datatypes.Set = datatypes.NewSet(1, 2, 3)
var a int = datatypes.Choose(func(x int) bool {
		return x >= 1
	}, S).(int) // a == 1
var b int = datatypes.Choose(func(y int) bool {
		return y < 0
	}, S).(int) // panic (no element satisfies predicate)
\end{lstlisting}
\captionof{figure}{Compiled Go}
\end{minipage}

\subsection{With}
The PlusCal \texttt{with} statement has the syntax

\noindent
\begin{minipage}[t]{\textwidth}
\begin{lstlisting}[language=pcal]
variables S_1 = {1, 2, 3}, a = "foo";
\* ...
with (x_1 \in S_1, x_2 = a, ...)
{
	\* do stuff with the x_i ...
}
\end{lstlisting}
\captionof{figure}{PlusCal}
\end{minipage}

This construct selects a random element from each \texttt{S\_i} and assigns them to the local variables \texttt{x\_i}. If the syntax \texttt{x\_i = a} is used, this simply assigns \texttt{a} to \texttt{x\_i}. In Go, this translates to

\noindent
\begin{minipage}[t]{\textwidth}
\begin{lstlisting}[language=golang]
import (
	"math/rand"
	"pgo/datatypes"
	"time"
)

var S_1 datatypes.Set = datatypes.NewSet(1, 2, 3)
var a string = "foo"
// ...
func main() {
	rand.Seed(time.Now().UnixNano())
	// ...
	{
		var x_1 int = S_1.ToSlice()[rand.Intn(S_1.Size())].(int)
		var x_2 string = a
		// ...
		// do stuff with the x_i ...
	}
}
\end{lstlisting}
\captionof{figure}{Compiled Go}
\end{minipage}

In Go the random number generator is seeded with the current Unix time at the beginning of the \texttt{main} function. The local variables declared by the \texttt{with} and its body are contained in a separate code block, to preserve the variables' local scope.

\subsection{Processes}
The PlusCal algorithm body can either contain statements in a uniprocess algorithm, or process declarations in a multiprocess algorithm. Processes can be declared with the syntax \verb|process (Name \in S)| or \verb|process (Name = Id)|. The first construct spawns a set of processes, one for each ID in the set \texttt{S}, and the second spawns a single process with ID \texttt{Id}. A process can refer to its identifier with the keyword \texttt{self}. In TLC, multiprocess algorithms are simulated by repeatedly selecting a random process then advancing it one atomic step (if it does not block).

PGo converts each process body to a function and spawns a goroutine per process. The function takes a single parameter \texttt{self}, the process ID. There are two semantic considerations: the main goroutine should not exit before all goroutines finish executing, and the time at which all child goroutines begin executing should be synchronized. To preserve these semantics, PGo uses a global waitgroup which waits on all goroutines, and each process body pulls from a dummy channel before beginning execution. When all goroutines have been initialized, the dummy channel is closed so that the channel pull no longer blocks, allowing for a synchronized start.

The following is a simple example:

\noindent
\begin{minipage}[t]{\textwidth}
\begin{lstlisting}[language=pcal]
variables
	idSet = {1, 2, 3};
	id = "id";
	
process (PName \in idSet)
	variable local;
{
	local := self;
}

process (Other = id) {
	print self;
}
\end{lstlisting}
\captionof{figure}{PlusCal}
\end{minipage}

\noindent
\begin{minipage}[t]{\textwidth}
\begin{lstlisting}[language=golang]
import (
	"fmt"
	"pgo/datatypes"
	"sync"
)

var idSet datatypes.Set = datatypes.NewSet(1, 2, 3) // PGo inferred type set[int]
var id string = "id" // PGo inferred type string
var PGoWait sync.WaitGroup
var PGoStart chan bool = make(chan bool)

func PName(self int)  {
	var local int // PGo inferred type int
	defer PGoWait.Done()
	<-PGoStart
	local = self
}
func Other(self string)  {
	defer PGoWait.Done()
	<-PGoStart
	fmt.Printf("%v\n", self)
}

func main()  {
	// spawn goroutines for PName processes
	for procId := range idSet.Iter() {
		PGoWait.Add(1)
		go PName(procId.(int))
	}
	PGoWait.Add(1)
	close(PGoStart)
	PGoWait.Wait()
}
\end{lstlisting}
\captionof{figure}{Compiled Go}
\end{minipage}

Note that all processes use the same waitgroup (\texttt{PGoWait}) and dummy channel (\texttt{PGoStart}).

\subsection{Labels}
\label{sec:labels}
In PlusCal, labels are used as targets for \texttt{goto} statements and also to specify atomic operations. If a statement is labeled, all statements until the next label are considered to be a single atomic operation. In Go, unused labels cause compile errors so PGo only inserts labels when they are used in \texttt{goto} statements.

To deal with atomicity, PGo divides the global variables into groups and guards each group with a \texttt{sync.RWMutex}. PGo groups variables by performing a disjoint-set union, merging two variable sets when two variables in them can be accessed in the same label. The following is a simple example:

\begin{minipage}[t]{\textwidth}
\begin{lstlisting}[language=pcal]
variables a = 0, b = 1, c = 2, d = 3;
process (P \in {1, 2, 3}) {
lab1:   a := 1;
        b := 2;
lab2:   b := 3;
        c := 4;
lab3:   d := 5;
}
\end{lstlisting}
\captionof{figure}{PlusCal}
\end{minipage}

\begin{minipage}[t]{\textwidth}
\begin{lstlisting}[language=golang]
var PGoLock []sync.RWMutex = make([]sync.RWMutex, 2)
var a int = 0 // PGo inferred type int; Lock group 0
var b int = 1 // PGo inferred type int; Lock group 0
var c int = 2 // PGo inferred type int; Lock group 0
var d int = 3 // PGo inferred type int; Lock group 1
var PGoWait sync.WaitGroup
var PGoStart chan bool = make(chan bool)

func P(self int)  {
	defer PGoWait.Done()
	<-PGoStart
	PGoLock[0].Lock()
	a = 1
	b = 2
	PGoLock[0].Unlock()
	PGoLock[0].Lock()
	b = 3
	c = 4
	PGoLock[0].Unlock()
	PGoLock[1].Lock()
	d = 5
	PGoLock[1].Unlock()
}
// ...
\end{lstlisting}
\captionof{figure}{Compiled Go}
\end{minipage}
The variable \texttt{b} may be accessed atomically with \texttt{a} (in the label \texttt{lab1}) and also with \texttt{c} (in the label \texttt{lab2}) so all three of \texttt{a}, \texttt{b}, and \texttt{c} must be grouped together to prevent data races. PGo locks the correct group before each atomic operation and unlocks it afterwards. Even single operations must use the lock, since there are no atomicity guarantees for most Go statements. If the atomic operations are specified to be smaller in PlusCal by adding more labels, PGo will compile smaller variable groups, allowing for more parallelism.

Locking may be turned off with the annotation \texttt{lock false}, and it can similarly be enforced using the annotation \texttt{lock true}.

\subsection{Limitations}
Not all PlusCal specifications can be compiled by PGo. This is an overview of some important PlusCal features that are currently unsupported.

\begin{itemize}
\item The \texttt{await} keyword

\item The \texttt{either} construct

\item Referencing \texttt{self} in a procedure call

\item TLA+ features:

	\begin{itemize}
	\item Alignment of boolean operators in bulleted lists determining precedence
	
	\item Records (structs)
	
	\item Bags (multisets)
	
	\item \texttt{IF ... THEN ... ELSE} and \texttt{CASE} expressions
	
	\item The \texttt{LET .. IN} construct
	
	\item Temporal logic operators
	
	\item Recursive definitions
	
	\item Operator definition (use definitions with parameters instead)
	\end{itemize}
\end{itemize}

The following are features that may be desirable for a developer working with the compiled program but are not yet considered by PGo:

\begin{itemize}
\item Support for networking

\item Interfacing with other programs

\item Reading input from sources other than the command line
\end{itemize}
